\noindent The Dirac equation $\left( i \gamma^{\mu} \partial_{\mu} - m \right) \psi = 0$ is motivated for us by the question of the dynamics of fermionic fields, espeically electrons. We introduce a new representation of the Lorentz group with intrinsic angular momentum by producing the gamma matrices $\gamma^{\mu}$, such that the anticommutation relation is satisfied. 

\begin{equation}
\{ \gamma^{\mu}, \, \gamma^{\nu} \} = \eta^{\mu\nu} \cdot \mathbb{I}
\end{equation}

\noindent And recall that the gamma matrices contain the Pauli spin matrices and have the form

\begin{equation}
\gamma^0 = \left( \begin{array}{c|c} 0 & \mathbb{I} \\ \hline \mathbb{I} & 0 \end{array} \right) \,\,\,\,\,\,\,\, \gamma^j = \left( \begin{array}{c|c} 0 & \sigma^j \\ \hline -\sigma^j & 0 \end{array} \right) .
\end{equation}

\noindent This will yield a new Lorentz-invariant equation of motion, and the steps to get there include

\begin{enumerate}
\item Calculating the Hamiltonian denisty from the Lagrangian density and guessing a quantization.
\item Entering the zero-mass limit, where the equation of motion decouples.
\item Solve the wave equation in the free particle basis.
\end{enumerate}

\noindent Recall that the Dirac spinor Lorentz-transforms according to 

\begin{equation}
\psi(x) \rightarrow \Lambda_{1/2} \psi (\Lambda^{-1} x)
\end{equation}

\noindent To build the Lagrangian density, we need to define some Lorentz scalars. A first guess may be to (incorrectly) define the Lorentz scalar as the quantity $\psi^\dagger (x) \psi (x)$, which Lorentz transforms as

\begin{equation}
\psi^\dagger (x) \psi (x) \rightarrow \psi^\dagger \Lambda_{1/2} ^\dagger \Lambda_{1/2} \psi
\end{equation}

\noindent The issue with this definition is that the representation of the Lorentz group can be unitary if it is finite dimensional, and, in general, $\Lambda_{1/2} ^\dagger \Lambda_{1/2}  \ne \mathbb{I}$, and $\Lambda_{1/2}$ is not unitary. The proper definition of the Lorentz scalar includes a gamma matrix factor to ensure unitarity. \\

\noindent  Define the Lorentz scalar as $\psi^\dagger (x) \gamma ^0 \psi (x)$, and define the quantity $\bar{\psi}(x) = \psi^\dagger (x) \gamma ^0$ to assist in proving that $\bar{\psi}\psi$ is indeed a Lorentz scalar.  \\

\noindent \textbf{Exercise}: Prove this using the \\

\textbf{Lemma}: $\Lambda_{1/2} ^\dagger \gamma^0 = \gamma^0 \Lambda_{1/2} ^{-1}$. \\

\noindent Similarly, define the Lorentz vector $v^\mu = \bar{\psi} \gamma^\mu \psi$. \\

\noindent Therefore, the Lagrangian density for the Dirac field is

\begin{equation}
\mathcal{L}_{Dirac} = \bar{\psi} \left( i \gamma^\mu \partial_\mu - m \right) \psi
\end{equation}

\noindent \textbf{Exercise}: Prove that $\mathcal{L}_{Dirac}$ is Lorentz scalar. \\
\noindent \textbf{Exercise}: Apply the Euler-Lagrange equation to obtain the Dirac equation. \\

\subsection*{Dirac Field Hamiltonian}

\noindent The conjugate momentum to the Dirac spinor $\psi$ is the quantity $i \psi^\dagger$, yielding the Hamiltonian

\begin{align}
H &= \int d^3 x \,\, \bar{\psi} \left( -i \underline{\gamma} \cdot \underline{\nabla} + m \right) \psi \\
H &= \int d^3 x \,\, \psi^\dagger \left( -i \gamma^0 \underline{\gamma} \cdot \underline{\nabla} + m \gamma^0 \right) \psi
\end{align}

\noindent Where the underlined vectors are only the spatial components, such that 
$$\underline{\gamma} \cdot \underline{\nabla}  = \sum_{j=1}^3 \gamma^j \partial_j.$$

\noindent And we have the dynamics of a classical Dirac field.

\subsection*{Zero-mass Limit}

\noindent Let's see how the Dirac equation and the four-component Dirac spinor decouples in the zero-mass limit to the two-component \textit{Weyl spinors} $\psi_{L/R}(x) \in \mathbb{C}^2$.

\begin{equation}
\psi(x) = \left( \begin{array}{c} \psi_1(x) \\ \psi_2(x) \\ \psi_3(x) \\ \psi_4(x) \end{array} \right) \rightarrow \left( \begin{array}{c}\psi_L(x) \\ \psi_R(x) \end{array} \right)
\end{equation}

\noindent Under the Lorentz transformation, Weyl spinors transform to themselves

\begin{align}
\psi_L &\rightarrow (\mathbb{I} - i \underline{\theta} \cdot \frac{1}{2} \underline{\sigma} - \underline{\beta} \cdot \frac{1}{2} \underline{\sigma} ) \psi_L \\
\psi_R &\rightarrow(\mathbb{I} - i \underline{\theta} \cdot \frac{1}{2} \underline{\sigma} + \underline{\beta} \cdot \frac{1}{2} \underline{\sigma} ) \psi_R
\end{align}

\noindent Where the rotation and boost three-vectors are, respectively, $\underline{\theta} = (\theta_1,\theta_2,\theta_3)$ and $\underline{\beta} = (\beta_1,\beta_2,\beta_3)$. \\

\noindent The left-handed spinor $\psi_L$ contains the right-handed spinor $\psi_R$, seen by taking the complex conjugate and multiplying by $\sigma^2$, since this causes a \textit{spin flip}, such that multiplication by $\sigma^2$ give the antipode on the Bloch sphere.

\begin{equation}
\sigma^2 \underline{\sigma}^* = - \underline{\sigma} \sigma^2
\end{equation}

\noindent Therefore, the quantity $\sigma^2 \psi_L^*$ transforms like $\psi_R$. \\

\noindent The coupling of the Weyl spinors is represented by the matrix equation (\textbf{Exercise})

\begin{equation}
\left( i \gamma^\mu \partial_\mu - m \right) \psi = \left( \begin{array}{cc} -m & i(\partial_0 + \underline{\sigma} \cdot \underline{\nabla}) \\ i(\partial_0 - \underline{\sigma} \cdot \underline{\nabla}) & m \end{array} \right) \left( \begin{array}{c} \psi_L \\ \psi_R \end{array} \right) = 0
\end{equation}

\noindent Now, suppose that $m=0$, such that $\psi_L$ and $\psi_R$ become two independent solutions to the Dirac equation, and we obtain a represntation of the Poincar\'e group. The equations of motion in the zero-mass limit are then

\begin{align}
i (\partial_0 - \underline{\sigma} \cdot \underline{\nabla} ) \psi_L &= 0 \\
i (\partial_0 + \underline{\sigma} \cdot \underline{\nabla} ) \psi_R &= 0 
\end{align}

\noindent Define the notation

\begin{align}
\sigma^\mu &= (1, \underline{\sigma}) \\
\bar{\sigma}^\mu &= (1, -\underline{\sigma})
\end{align}

\noindent Then the gamma matrices become

\begin{equation}
\gamma^\mu = \left( \begin{array}{cc} 0 & \sigma^\mu \\ \bar{\sigma}^\mu & 0 \end{array} \right)
\end{equation}

\noindent And we rewrite the Dirac equations of motion as

\begin{align}
i (\bar{\sigma} \cdot \partial) \psi_L &= 0 \\
i (\sigma \cdot \partial) \psi_R &= 0 
\end{align}

\subsection*{Free Particle Solutions to the Dirac Equation}

\noindent The Dirac equation is a linear system of partial differental equations (PDEs), and the solutions may be expressed in terms of plane waves

\begin{equation}
\psi(x) = e^{-i p \cdot x} u(p)
\end{equation}

\noindent Where $u(p) \in \mathbb{C}^4$ and are the zero eigenvectors of the Dirac matrix. Linear combinations of $\psi(x)$ yield a general solution, with the constraint of \textit{on-shell momentum}, such that $p^2 = m^2, \, p^0 > 0$. \\

\noindent Substitute the plane wave solution in to Dirac equation to obtain the matrix equation

\begin{equation}
(p_\mu \gamma^\mu - m \cdot \mathbb{I}) \, u(p) = 0
\end{equation}

\noindent Now, solving for $u(p)$, enter the rest frame, where $p_0 = (m,0)$, and the matrix equation becomes

\begin{equation}
\left( m \left( \begin{array}{cc} 0 & \mathbb{I} \\ \mathbb{I} & 0 \end{array} \right) - m \left( \begin{array}{cc} \mathbb{I} & 0 \\ 0 & \mathbb{I} \end{array} \right)  \right) \left( \begin{array}{c} u_L(p_0) \\ u_R(p_0) \end{array} \right) = 0
\end{equation}

\noindent And solutions are of the form

\begin{equation}
u_L(p_0) = u_R(p_0) = \sqrt{m} \, \xi
\end{equation}

\noindent Where $\xi$ fixed, with normalization $\xi^\dagger \xi = 1$. Then the solution in the rest frame is 

\begin{equation}
\psi^s(x) = e^{-i p \cdot x} u^s(p_0) = \sqrt{m} \left(\begin{array}{c} \xi^s \\ \xi^s \end{array} \right) e^{-i p \cdot x}
\end{equation}

\noindent Where $s=1,2$, and we may write $\xi^1 = \left( \begin{array}{c} 1 \\ 0 \end{array} \right)$ and $\xi^2 = \left( \begin{array}{c} 0 \\ 1 \end{array} \right)$.

\noindent The solution in a general reference frame is obtained via the spinor representation of the Lorentz boost $\Lambda$. To calculate the spinor representation of the Lorentz boost along any three-direction, recall that, for rapidity $\eta$

\begin{align}
\Lambda &= \cosh (\eta) \cdot \mathbb{I} + \sinh(\eta) \cdot \left( \begin{array}{cccc} 0&0&0&1\\0&0&0&0\\0&0&0&0\\1&0&0&0 \end{array} \right) \\
&= e^{-i \frac{1}{2} \omega_{\mu\nu} J^{\mu\nu}} \\
\Lambda &= e^{-i \frac{1}{2} \omega_{0 3} J^{0 3} }
\end{align}

\noindent Where $J^{\mu\nu}$ obey the Lie algebra of the Lorentz group, $\omega_{\mu\nu}$ are the 6 boost parameters, and $\omega_{0 3}=2\eta$ is the only nonzero term (\textbf{Exercise}).

\noindent To apply this transformation on the spinor, apply the matrix

\begin{align}
\Lambda_{1/2} &= e^{-\frac{i}{2} \omega_{03} S^{03} }\\
&= e^{-\frac{i}{2} \cdot 2 \eta \cdot \frac{i}{4} [ \gamma^0, \, \gamma^3 ]} \\
&= \left( \begin{array}{cc} e^{-\frac{1}{2} \eta \sigma^3} & 0 \\ 0 & e^{\frac{1}{2} \eta \sigma^3} \end{array} \right) 
\end{align}

\noindent So, in the general reference frame, $p=\Lambda p_0$, the solutions have the form (\textbf{Exercise})

\begin{align}
\psi(x) &= e^{-i p \cdot x} u^s(p=\Lambda p_0) \\
&= \sqrt{m} \left( \begin{array}{c} e^{-\frac{1}{2} \eta \sigma^3} \xi^s \\ e^{\frac{1}{2} \eta \sigma^3} \xi^s \end{array} \right) e^{-i p \cdot x} \\
&= \left( \begin{array}{c} \sqrt{p\cdot \sigma} \xi^s \\ \sqrt{p \cdot \bar{\sigma}} \xi^s \end{array} \right) e^{-i p \cdot x}
\end{align}

\noindent Where we used $(p \cdot \sigma)(p \cdot \bar{\sigma}) = m^2 = p^2$, and we have two linearly independent solutions to the Dirac equation, but we need four solutions per given momentum. 

\noindent The other two solutions come from the \textit{plane wave ansatz} for the negative frequency solutions.

\begin{equation}
\psi_{negative}(x) = v(p) e^{i p \cdot x}
\end{equation}

\noindent Similar to the positive frequency case, there are two linearly independent solutions

\begin{align}
\psi_{negative} (x) &= e^{i p \cdot x} v^s (p) \\
&= \left( \begin{array}{c} \sqrt{p\cdot \sigma} \xi^s \\ -\sqrt{p \cdot \bar{\sigma}} \xi^s \end{array} \right) e^{i p \cdot x}
\end{align}

\noindent To normalize these states, recall that $\bar{\psi}\psi$ is Lorentz invariant, with $\psi(x) = u^s(p) e^{-ip \cdot x}$, such that 

\begin{equation}
\bar{\psi} \psi = \bar{u}^s (p) u^s (p) = 2 m (\xi^s)^\dagger \xi^s
\end{equation}

\noindent And more generally, for both positive and negative frequency solutions

\begin{align}
\bar{\psi} \psi &= \bar{u}^r (p) u^s (p) = 2 m \delta^{rs} \\
\bar{\psi}_{negative} \psi_{negative} &= \bar{v}^r (p) v^s (p) = -2 m \delta^{rs}
\end{align}

\noindent And for mixed positive-negative frequency solutions, the normalization is zero

\begin{equation}
\bar{u}^r(p) v^s (p) = \bar{v}^r (p) u^s (p) = 0
\end{equation}

\noindent There we have four linearly independent solutions to the (classical) Dirac equation in a general reference frame.